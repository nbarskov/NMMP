\documentclass[a4paper,14pt,russian, leqno, fleqn]{extreport}
\usepackage{extsizes}
\usepackage{cmap} % для кодировки шрифтов в pdf
\usepackage[T2A]{fontenc}
\usepackage[utf8]{inputenc}
\usepackage[russian]{babel}
\usepackage[left=25mm, top=20mm, right=10mm, bottom=20mm, nohead, nofoot]{geometry}
\usepackage{graphicx} % для вставки картинок
\usepackage{amssymb,amsfonts,amsmath,amsthm} % математические дополнения от АМС
\usepackage{indentfirst} % отделять первую строку раздела абзацным отступом тоже
\usepackage[usenames,dvipsnames]{color} % названия цветов
\usepackage{makecell}
\usepackage{multirow} % улучшенное форматирование таблиц
\usepackage{ulem} % подчеркивания
%\usepackage{pscyr}
%\usepackage{txfonts}
\usepackage{longtable}
\linespread{1.3} % полуторный интервал
%\renewcommand{\rmdefault}{ftm} % Times New Roman
\frenchspacing

%Нумерация страниц
\usepackage{fancyhdr}
\pagestyle{fancy}
\fancyhf{}
\fancyhead[R]{\thepage}
\fancyheadoffset{0mm}
\fancyfootoffset{0mm}
\setlength{\headheight}{17pt}
\renewcommand{\headrulewidth}{0pt}
\renewcommand{\footrulewidth}{0pt}
\fancypagestyle{plain}{ 
	\fancyhf{}
	\rhead{\thepage}}
\setcounter{page}{5} % начать нумерацию страниц с №5

%Подписи таблиц
\usepackage[tableposition=top]{caption}
\usepackage{subcaption}abc
\DeclareCaptionLabelFormat{gostfigure}{Рисунок #2}
\DeclareCaptionLabelFormat{gosttable}{Таблица #2}
\DeclareCaptionLabelSeparator{gost}{~---~}
\captionsetup{labelsep=gost}
\captionsetup[figure]{labelformat=gostfigure}
\captionsetup[table]{labelformat=gosttable}
\renewcommand{\thesubfigure}{\asbuk{subfigure}}

%Отступы
\usepackage{titlesec}

\titleformat{\chapter}[display]
{\filcenter}
{\MakeUppercase{\chaptertitlename} \thechapter}
{8pt}
{\bfseries}{}

\titleformat{\section}
{\normalsize\bfseries}
{\thesection}
{1em}{}

\titleformat{\subsection}
{\normalsize\bfseries}
{\thesubsection}
{1em}{}

% Настройка вертикальных и горизонтальных отступов
\titlespacing*{\chapter}{0pt}{-30pt}{8pt}
\titlespacing*{\section}{\parindent}{*5}{*5}
\titlespacing*{\subsection}{\parindent}{*8}{*8}

%Списки
\usepackage{enumitem}
\makeatletter
\AddEnumerateCounter{\asbuk}{\@asbuk}{м)}
\makeatother
\setlist{nolistsep}
\renewcommand{\labelitemi}{-}
\renewcommand{\labelenumi}{\asbuk{enumi})}
\renewcommand{\labelenumii}{\arabic{enumii})}

%Оглавление
\usepackage{tocloft}
\renewcommand{\cfttoctitlefont}{\hspace{0.38\textwidth} \bfseries\MakeUppercase}
\renewcommand{\cftbeforetoctitleskip}{-1em}
\renewcommand{\cftaftertoctitle}{\mbox{}\hfill \\ \mbox{}\hfill{\footnotesize Стр.}\vspace{-2.5em}}
\renewcommand{\cftchapfont}{\normalsize\bfseries \MakeUppercase{\chaptername} }
\renewcommand{\cftsecfont}{\hspace{31pt}}
\renewcommand{\cftsubsecfont}{\hspace{11pt}}
\renewcommand{\cftbeforechapskip}{1em}
\renewcommand{\cftparskip}{-1mm}
\renewcommand{\cftdotsep}{1}
\setcounter{tocdepth}{2} % задать глубину оглавления — до subsection включительно

\newcommand{\empline}{\mbox{}\newline}
\newcommand{\likechapterheading}[1]{ 
	\begin{center}
		\textbf{\MakeUppercase{#1}}
	\end{center}
	\empline}

\makeatletter
\renewcommand{\@dotsep}{2}
\newcommand{\l@likechapter}[2]{{\bfseries\@dottedtocline{0}{5pt}{5pt}{#1}{#2}}}
\makeatother
\newcommand{\likechapter}[1]{    
	\likechapterheading{#1}    
	\addcontentsline{toc}{likechapter}{\MakeUppercase{#1}}}


\begin{document}
	
	\tableofcontents
	
	\chapter{Лабораторная работа № 1}
	\section{Тема работы. Цель работы. Порядок выполнения работы}
	\textbf{Тема:} решение краевых задач математической физики методом конечных разностей.
	
	\textbf{Цель работы:} получение практических навыков построения и исследования разностных схем для задач математической физики, разработки вычислительных алгоритмов и компьютерных программ для их решения.
	
	\textbf{Порядок выполнения лабораторной работы:}
	\begin{enumerate}
		\item осуществить математическую постановку краевой задачи для физического процесса, описанного в предложенном варианте работы.
		\item осуществить построение разностной схемы, приближающей полученную краевую задачу. При этом следует согласовать с преподавателем тип разностной схемы;
		\item провести теоретическое исследование схемы: показать, что схема аппроксимирует исходную краевую задачу, и найти порядки аппроксимации относительно шагов дискретизации; исследовать устойчивость схемы и сходимость сеточного решения к решению исходной задачи математической физики;
		\item разработать алгоритм численного решения разностной краевой задачи;
		\item разработать компьютерную программу, реализующую созданный алгоритм, с интерфейсом, обеспечивающим следующие возможности: диалоговый режим ввода физических, геометрических и сеточных параметров задачи; графическую визуализацию численного решения задачи;
		\item провести исследование зависимости численного решения от величин параметров дискретизации;
		\item оформить отчет о проделанной работе в соответствии с требованиями, изложенными в методических указаниях.
	\end{enumerate}
	
	\section{Индивидуальное задание на лабораторную работу}	
	Разработать программу численного моделирования процесса остывания тонкой однородной пластины, имеющей форму диска радиусом $R$ и толщиной $l$. Между гранями пластины и окружающей средой, имеющей температуру $u_c$, происходит теплообмен, описываемый законом Ньютона с коэффициентом теплообмена $\alpha$.  На боковой поверхности $r=R$ пластины поддерживается температура $u_b$. В начальный момент времени поле пластины обладает осевой симметрией, т.е. распределение температуры по пластине зависит только от радиальной координаты $r$ полярной системы, т.е. $u\lvert_{t=0} = \psi(r), 0 \leq r \leq R$.
	
	Пластина выполнена из материала, характеризуемого коэффициентами теплопроводности $k$, объемной теплоемкости $с$.
	
	Для численного решения задачи теплопроводности на временном промежутке  использовать:
	
	\begin{enumerate}
		\item Простейшую явную конечно-разностную схему;
		\item Простейшую неявную конечно-разностную схему;
	\end{enumerate}
	
	При проведении расчетов использовать значения параметров $R$, $l$, $u_c$, $u_b$, $\alpha$, $T$, $k$, $c$ и выражение функции $\psi(r)$, указанные преподавателем.
	
	\begin{longtable}[c]{|l|l|}
		\caption{Значения параметров задачи}\\
			\hline
			Радиус пластины $R$ & $6$ см \\
			\hline
			Толщина пластины $l$ & $0,5$ см\\
			\hline
			Температура окружающей среды $u_c$ & $20^{\circ}$ С\\
			\hline
			Коэффициент теплообмена $\alpha$ & $0,002 \ \dfrac{\strut\text{Вт}}{\strut\text{см}^2 \cdot \text{К}}$\\
			\hline
			Температура боковой поверхности $u_b$ & $(20 + \psi(r))^{\circ}$С \\
			\hline
			Продолжительность эксперимента $T$ & $50$ c \\
			\hline
			Коэффициент теплопроводности $k$ & $0,59 \ \dfrac{\strut\text{Вт}}{\strut\text{см} \cdot \text{К}}$ \\
			\hline
			Объемная теплоемкость $c$ & $1,65 \ \dfrac{\strut\text{Дж}}{\strut\text{см}^3 \cdot \text{К}}$ \\
			\hline	
	\end{longtable}

	\section{Описание этапов выполнения лабораторной работы}
	\subsection{Математическая поставновка задачи}
	Процесс, происходящий в пластинке, описывается уравнением теплопроводности:
	\begin{equation}
	\dfrac{\partial u}{\partial t} = \dfrac{k}{c}\nabla^2u + f(r,t)
	\end{equation}
	
	В постановке задачи указано, что начальное распределение имеет радиальную симметрию. Учтем это и заключим, что дальнейший процесс так же имеет радиальную симметрию, т.е 
	\begin{equation}
	u = u(r,t), \quad u(\varphi) = const	
	\end{equation}
	
	тогда:	
	\begin{equation}
	\dfrac{\partial u}{\partial t} = \dfrac{k}{c} \left ( \dfrac{1}{r}\dfrac{\partial u}{\partial r} + \dfrac{\partial^2 u}{\partial r^2}\right ) + f(r,t)
	\label{eq: ut}
	\end{equation}
	
	Теплообмен между поверхностью пластинки и окружающей средой согласно условию задачи описывается при помощи закона Ньютона в дифференциальной форме:	
	\begin{equation}
	\dfrac{d}{dt}\dfrac{\partial Q}{\partial S} = \alpha \Delta u  = \alpha (u_c - u)
	\end{equation}
	
	Если считать, что температура окружающей среды однородна, то:	
	\begin{equation}
	\dfrac{\partial u}{\partial t} = \dfrac{\alpha S}{cSl}(u_c - u) = \dfrac{\alpha}{cl}(u_c - u) = f(r,t)
	\label{eq: f}
	\end{equation}
	
	Особо следует рассмотреть граничное условие в точке $r = 0$. Используя соотношение (23) из методических указаний найдем:	
	\begin{equation}
	\dfrac{\partial u}{\partial t} = \dfrac{2k}{c} \dfrac{\partial^2 u}{\partial r^2} + \dfrac{\alpha}{cl}(u_c - u)
	\label{eq: utr0}
	\end{equation}
	
	Тогда, учитывая выражения \eqref{eq: ut}, \eqref{eq: f} и \eqref{eq: utr0}, а также условие задачи, постановка задачи Коши выглядит следующим образом:	
	\begin{equation}
	\left \{
	\begin{array}{l}
	\dfrac{\partial u}{\partial t} = a^2 \left ( \dfrac{1}{r}\dfrac{\partial u}{\partial r} + \dfrac{\partial^2 u}{\partial r^2}\right ) + \dfrac{\alpha}{cl}(u_c - u), 0 < r \leq R\\
	\\
	\dfrac{\partial u}{\partial t} = \dfrac{2k}{c} \dfrac{\partial^2 u}{\partial r^2} + \dfrac{\alpha}{cl}(u_c - u), r = 0 \\
	\\
	u(0,r) = \psi(r), 0 \leq r \leq R \\
	\\
	u(t, R) = u_b + J_0\left(\dfrac{\mu r}{R}\right), 0 \leq t \leq T
	\end{array}
	\right.
	\label{eq: koschi_task}
	\end{equation}

	\subsection{Построение аналитического решения}
	Рассмотрим систему \eqref{eq: koschi_task} в следующем виде:
	\begin{equation}
		\left \{\begin{array}{l}
		\dfrac{\partial u}{\partial t} = a^2 \left ( \dfrac{1}{r}\dfrac{\partial u}{\partial r} + \dfrac{\partial^2 u}{\partial r^2}\right ) + \dfrac{\alpha}{cl}(u_c - u), 0 < r \leq R\\
		\\
		u(0, r) = u_b + J_0\left(\dfrac{\mu r}{R}\right), 0 \leq r \leq R
		\end{array}\right.
	\end{equation}
	
	Преобразуем правую часть первого уравнения этой системы:
	\begin{equation}
		a^2 \left ( \dfrac{1}{r}\dfrac{\partial u}{\partial r} + \dfrac{\partial^2 u}{\partial r^2}\right ) - \dfrac{\alpha}{cl}(u - u_c) \Rightarrow \dfrac{1}{r}\dfrac{\partial u}{\partial r} + \dfrac{\partial^2 u}{\partial r^2} - \beta (u - u_c)
	\end{equation}	
	
	Осуществим замену:
	
	Таким образом правая часть представляет собой оператор Штурма-Лиувилля, для которого можно решить спектральную задачу:
	\begin{equation}
		\hat{L}v = \lambda v
	\end{equation}
	\begin{equation}
		\dfrac{1}{r}\dfrac{\partial v}{\partial r} + \dfrac{\partial^2 v}{\partial r^2} - \beta v = \lambda v
	\end{equation}
	Это можно привести к уравнению Бесселя:
	
	\subsection{Построение разностной схемы}
	
	Для построения разностной схемы для начала введем равномерную сетку с шагами $h_r$ --- пространственный шаг и $h_t$ --- шаг по времени:	
	\begin{equation}
	\begin{array}{l}
	r_i = ih_r, \quad i = \overline{0,I} \quad \\                         
	t_k = kh_t, \quad k = \overline{0,K}
	\end{array}
	\label{eq:grid}
	\end{equation}
	
	Значения шагов равномерной сетки можно найти из следующих соотношений:	
	\begin{equation}
	h_r = \dfrac{R}{I}
	\end{equation}	
	\begin{equation}
	h_t = \dfrac{T}{K}
	\end{equation}

	И рассмотрим 4 разностных отношения для производных по времени и пространству:	
	\begin{equation}
	\left.\dfrac{\partial u}{\partial t}\right|_{(r_i,t_k)} = \dfrac{u_i^k-u_i^{k-1}}{h_t}, \quad i = \overline{0, I}, k = \overline{1,K}
	\label{eq: time_implicit}
	\end{equation}	
	\begin{equation}
	\left.\dfrac{\partial u}{\partial t}\right|_{(r_i,t_k)} = \dfrac{u_i^{k+1}-u_i^k}{h_t}, \quad i = \overline{0, I}, k = \overline{0,K}
	\label{eq: time_explicit}
	\end{equation}	
	\begin{equation}
	\left.\dfrac{\partial u}{\partial r}\right|_{(r_i, t_k)} = \dfrac{u_{i+1}^k - u_{i-1}^k}{2h_r}, \quad i = \overline{1,I-1}, k = \overline{0,K}
	\label{eq: first_d}
	\end{equation}	
	\begin{equation}
	\left.\dfrac{\partial^2 u}{\partial r^2}\right|_{(r_i, t_k)} = \dfrac{u_{i+1}^k - 2u_i^k + u_{i-1}^k}{h^2_r}, \quad i = \overline{1,I-1}, k = \overline{0,K}
	\label{eq: second_d}
	\end{equation}
	
	\paragraph{Построение неявной разностной схемы}
	
	При построении явной разностной схемы воспользуемся соотношениями \eqref{eq: time_explicit}, \eqref{eq: first_d}, \eqref{eq: second_d}:
	
	\begin{equation}
		\left\{\begin{array}{l}
		\dfrac{u_i^{k+1}-u_i^k}{h_t} = \alpha^2\left(\dfrac{1}{ih_r}\dfrac{u_{i+1}^k - u_{i-1}^k}{2h_r} +\dfrac{u_{i+1}^k - 2u_i^k + u_{i-1}^k}{h^2_r}\right) + \dfrac{\alpha}{cl}\left (u_c - u \right) \\
		i = \overline{1,I-1}, \quad k = \overline{0,K-1}	 \\
		\dfrac{u_0^{k+1}-u_0^k}{h_t} = \dfrac{2k}{c}\cdot\dfrac{u_1^k-2u_0^k+u_{-1}^k}{h_r^2} + \dfrac{\alpha}{cl}(u_c-u) \\
		u_i^0 = \psi(ih_r) \\
		i = \overline{0, I-1} \\
		u_I^k = u_b \\
		k = \overline{0,K}

	\end{array}\right.
	\end{equation}
	В связи с радиальной симметрией запишем соотношение для первой пространственной производной:
	\begin{equation}
		\dfrac{u_1 - u_{-1}}{h_r} = 0
	\end{equation}
	Из этого соотношения получим:
	\begin{equation}
		u_{-1} = u_1
	\end{equation}
	В соответствии с этим получим:
	\begin{equation}
		\left\{\begin{array}{l}
		\dfrac{u_i^{k+1}-u_i^k}{h_t} = \alpha^2\left(\dfrac{1}{ih_r}\dfrac{u_{i+1}^k - u_{i-1}^k}{2h_r} +\dfrac{u_{i+1}^k - 2u_i^k + u_{i-1}^k}{h^2_r}\right) + \dfrac{\alpha}{cl}\left (u_c - u \right) \\
		i = \overline{1,I-1}, \quad k = \overline{0,K-1}	 \\
		\dfrac{u_0^{k+1}-u_0^k}{h_t} = \dfrac{4k}{c}\cdot\dfrac{u_1^k-u_0^k}{h_r^2} + \dfrac{\alpha}{cl}(u_c-u_0^k) \\
		u_i^0 = \psi(ih_r) \\
		i = \overline{0, I-1} \\
		u_I^k = u_b \\
		k = \overline{0,K}
	\end{array}\right.
	\end{equation}

	\subsection{Исследование аппроксимации и устойчивости разностной схемы}
	
	\chapter{Лабораторная работа №2}
	\section{Тема работы. Цель работы. Порядок выполнения работы}
	\section{Описание этапов выполнения лабораторной работы}
	\subsection{Описание тестовой задачи, ее аналитическое решение}
	\subsection{Экспериментальное исследование фактической скорости сходимости сеточного решения к точному для тестовой задачи}
\end{document}