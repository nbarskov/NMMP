%\documentclass[a4paper,14pt,russian, leqno, fleqn]{extreport}
\documentclass[a4paper,14pt,russian, fleqn]{extreport}
\usepackage{extsizes}
\usepackage{cmap} % для кодировки шрифтов в pdf
\usepackage[T2A]{fontenc}
\usepackage[utf8]{inputenc}
\usepackage[russian]{babel}
\usepackage[left=25mm, top=20mm, right=10mm, bottom=20mm, nohead, nofoot]{geometry}
\usepackage{graphicx} % для вставки картинок
\usepackage{amssymb,amsfonts,amsmath,amsthm} % математические дополнения от АМС
\usepackage{indentfirst} % отделять первую строку раздела абзацным отступом тоже
\usepackage[usenames,dvipsnames]{color} % названия цветов
\usepackage{makecell}
\usepackage{multirow} % улучшенное форматирование таблиц
\usepackage{ulem} % подчеркивания
%\usepackage{pscyr}
%\usepackage{txfonts}
\usepackage{longtable}
\linespread{1.15} % полуторный интервал
%\renewcommand{\rmdefault}{ftm} % Times New Roman
\frenchspacing

%Нумерация страниц
\usepackage{fancyhdr}
\pagestyle{fancy}
\fancyhf{}
\fancyhead[R]{\thepage}
\fancyheadoffset{0mm}
\fancyfootoffset{0mm}
\setlength{\headheight}{17pt}
\renewcommand{\headrulewidth}{0pt}
\renewcommand{\footrulewidth}{0pt}
\fancypagestyle{plain}{ 
	\fancyhf{}
	\rhead{\thepage}}
\setcounter{page}{5} % начать нумерацию страниц с №5

%Подписи таблиц
\usepackage[tableposition=top]{caption}
\usepackage{subcaption}
\DeclareCaptionLabelFormat{gostfigure}{Рисунок #2}
\DeclareCaptionLabelFormat{gosttable}{Таблица #2}
\DeclareCaptionLabelSeparator{gost}{~---~}
\captionsetup{labelsep=gost}
\captionsetup[figure]{labelformat=gostfigure}
\captionsetup[table]{labelformat=gosttable}
\renewcommand{\thesubfigure}{\asbuk{subfigure}}

%Отступы
\usepackage{titlesec}

\titleformat{\chapter}[display]
{\filcenter}
{\MakeUppercase{\chaptertitlename} \thechapter}
{8pt}
{\bfseries}{}

\titleformat{\section}
{\normalsize\bfseries}
{\thesection}
{1em}{}

\titleformat{\subsection}
{\normalsize\bfseries}
{\thesubsection}
{1em}{}

% Настройка вертикальных и горизонтальных отступов
\titlespacing*{\chapter}{0pt}{-10pt}{5pt}
\titlespacing*{\section}{\parindent}{*3}{*1}
\titlespacing*{\subsection}{\parindent}{*3}{*1}

%Списки
\usepackage{enumitem}
\makeatletter
\AddEnumerateCounter{\asbuk}{\@asbuk}{м)}
\makeatother
\setlist{nolistsep}
\renewcommand{\labelitemi}{-}
\renewcommand{\labelenumi}{\asbuk{enumi})}
\renewcommand{\labelenumii}{\arabic{enumii})}

%Оглавление
\usepackage{tocloft}
\renewcommand{\cfttoctitlefont}{\hspace{0.38\textwidth} \bfseries\MakeUppercase}
\renewcommand{\cftbeforetoctitleskip}{-1em}
\renewcommand{\cftaftertoctitle}{\mbox{}\hfill \\ \mbox{}\hfill{\footnotesize Стр.}\vspace{-2.5em}}
\renewcommand{\cftchapfont}{\normalsize\bfseries \MakeUppercase{\chaptername} }
\renewcommand{\cftsecfont}{\hspace{31pt}}
\renewcommand{\cftsubsecfont}{\hspace{11pt}}
\renewcommand{\cftbeforechapskip}{1em}
\renewcommand{\cftparskip}{-1mm}
\renewcommand{\cftdotsep}{1}
\setcounter{tocdepth}{2} % задать глубину оглавления — до subsection включительно

\newcommand{\empline}{\mbox{}\newline}
\newcommand{\likechapterheading}[1]{ 
	\begin{center}
		\textbf{\MakeUppercase{#1}}
	\end{center}
	\empline}

\makeatletter
\renewcommand{\@dotsep}{2}
\newcommand{\l@likechapter}[2]{{\bfseries\@dottedtocline{0}{5pt}{5pt}{#1}{#2}}}
\makeatother
\newcommand{\likechapter}[1]{    
	\likechapterheading{#1}    
	\addcontentsline{toc}{likechapter}{\MakeUppercase{#1}}}


\begin{document}
	\begin{center}
		Федеральное государственное автономное образовательное учреждение \\
		высшего образования \\
		<<Самарский национальный исследовательский университет \\
		имени академика С. П. Королева>>\\
		\vspace{2cm}
		Факультет информатики \\
		\vspace{1cm}
		Кафедра технической кибернетики \\
		\vspace{3cm}
		\textbf{Отчет по лабораторным работам №1 и №2} \\
		по дисциплине <<Численные методы математической физики>> \\
		\vspace{1cm}
		Тема: \textbf{РЕШЕНИЕ КРАЕВЫХ ЗАДАЧ МАТЕМАТИЧЕСКОЙ ФИЗИКИ МЕТОДОМ КОНЕЧНЫХ РАЗНОСТЕЙ}\\
		\vspace{1cm}
		Вариант 9
	\end{center}

	\noindentВыполнил студент Барсков Н. М. \\
	Группа 6406\\
	Преподаватель Дегтярев А. А.
	\vfill
	\begin{center}
		2017
	\end{center}
	\newpage
	
	\tableofcontents
	
	\chapter{Лабораторная работа № 1}
	\section{Тема работы. Цель работы. Порядок выполнения работы}
	\textbf{Тема:} решение краевых задач математической физики методом конечных разностей.
	
	\textbf{Цель работы:} получение практических навыков построения и исследования разностных схем для задач математической физики, разработки вычислительных алгоритмов и компьютерных программ для их решения.
	
	\textbf{Порядок выполнения лабораторной работы:}
	\begin{enumerate}
		\item осуществить математическую постановку краевой задачи для физического процесса, описанного в предложенном варианте работы.
		\item осуществить построение разностной схемы, приближающей полученную краевую задачу. При этом следует согласовать с преподавателем тип разностной схемы;
		\item провести теоретическое исследование схемы: показать, что схема аппроксимирует исходную краевую задачу, и найти порядки аппроксимации относительно шагов дискретизации; исследовать устойчивость схемы и сходимость сеточного решения к решению исходной задачи математической физики;
		\item разработать алгоритм численного решения разностной краевой задачи;
		\item разработать компьютерную программу, реализующую созданный алгоритм, с интерфейсом, обеспечивающим следующие возможности: диалоговый режим ввода физических, геометрических и сеточных параметров задачи; графическую визуализацию численного решения задачи;
		\item провести исследование зависимости численного решения от величин параметров дискретизации;
		\item оформить отчет о проделанной работе в соответствии с требованиями, изложенными в методических указаниях.
	\end{enumerate}
	
	\section{Индивидуальное задание на лабораторную работу}	
	Разработать программу численного моделирования процесса остывания тонкой однородной пластины, имеющей форму диска радиусом $R$ и толщиной $l$. Между гранями пластины и окружающей средой, имеющей температуру $u_c$, происходит теплообмен, описываемый законом Ньютона с коэффициентом теплообмена $\alpha$.  На боковой поверхности $r=R$ пластины поддерживается температура $u_b$. В начальный момент времени поле пластины обладает осевой симметрией, т.е. распределение температуры по пластине зависит только от радиальной координаты $r$ полярной системы, т.е. $u\lvert_{t=0} = \psi(r), 0 \leq r \leq R$.
	
	Пластина выполнена из материала, характеризуемого коэффициентами теплопроводности $k$, объемной теплоемкости $с$.
	
	Для численного решения задачи теплопроводности на временном промежутке  использовать:
	
	\begin{enumerate}
		\item Простейшую явную конечно-разностную схему;
		\item Простейшую неявную конечно-разностную схему;
	\end{enumerate}
	
	При проведении расчетов использовать значения параметров $R$, $l$, $u_c$, $u_b$, $\alpha$, $T$, $k$, $c$ и выражение функции $\psi(r)$, указанные преподавателем.
	
	\begin{longtable}[c]{|l|l|}
		\caption{Значения параметров задачи}\\
			\hline
			Радиус пластины $R$ & $6$ см \\
			\hline
			Толщина пластины $l$ & $0,5$ см\\
			\hline
			Температура окружающей среды $u_c$ & $20^{\circ}$ С\\
			\hline
			Коэффициент теплообмена $\alpha$ & $0,002 \ \dfrac{\strut\text{Вт}}{\strut\text{см}^2 \cdot \text{К}}$\\
			\hline
			Температура боковой поверхности $u_b$ & $(20 + \psi(r))^{\circ}$С \\
			\hline
			Продолжительность эксперимента $T$ & $50$ c \\
			\hline
			Коэффициент теплопроводности $k$ & $0,59 \ \dfrac{\strut\text{Вт}}{\strut\text{см} \cdot \text{К}}$ \\
			\hline
			Объемная теплоемкость $c$ & $1,65 \ \dfrac{\strut\text{Дж}}{\strut\text{см}^3 \cdot \text{К}}$ \\
			\hline	
	\end{longtable}

	\section{Описание этапов выполнения лабораторной работы}
	\subsection{Математическая поставновка задачи}
	Процесс, происходящий в пластинке, описывается уравнением теплопроводности:
	\begin{equation*}
	\dfrac{\partial u}{\partial t} = \dfrac{k}{c}\nabla^2u + f(r,t)
	\end{equation*}
	В постановке задачи указано, что начальное распределение имеет радиальную симметрию. Учтем это и заключим, что дальнейший процесс так же имеет радиальную симметрию, т.е 
	\begin{equation*}
	u = u(r,t), \quad u(\varphi) = const	
	\end{equation*}
	тогда:	
	\begin{equation*}
	\dfrac{\partial u}{\partial t} = \dfrac{k}{c} \left ( \dfrac{1}{r}\dfrac{\partial u}{\partial r} + \dfrac{\partial^2 u}{\partial r^2}\right ) + f(r,t)
	\end{equation*}
	Теплообмен между поверхностью пластинки и окружающей средой согласно условию задачи описывается при помощи закона Ньютона в дифференциальной форме:	
	\begin{equation*}
	\dfrac{d}{dt}\dfrac{\partial Q}{\partial S} = \alpha \Delta u  = \alpha (u_b - u)
	\end{equation*}
	Если считать, что температура окружающей среды однородна, то:	
	\begin{equation*}
	\dfrac{\partial u}{\partial t} = \dfrac{\alpha S}{cSl}(u_b - u) = \dfrac{\alpha}{cl}(u_b - u) = f(r,t)	
	\end{equation*}
	Особо следует рассмотреть граничное условие в точке $r = 0$. Используя соотношение (23) из методических указаний найдем:	
	\begin{equation*}
	\dfrac{\partial u}{\partial t} = \dfrac{2k}{c} \dfrac{\partial^2 u}{\partial r^2} + \dfrac{\alpha}{cl}(u_b - u)
	\end{equation*}
	Тогда, учитывая вышеперечисленные выражения, а также условие задачи, постановка задачи Коши выглядит следующим образом:	
	\begin{equation}\label{KoshiTask}
	\left \{
	\begin{array}{l}
	\dfrac{\strut \partial u}{\strut \partial t} = a^2 \left ( \dfrac{\strut 1}{\strut r}\dfrac{\strut \partial u}{\strut \partial r} + \dfrac{\strut \partial^2 u}{\strut \partial r^2}\right ) + \dfrac{\strut \alpha}{\strut cl}(u_b - u), 0 < r \leq R\\
	\dfrac{\strut \partial u}{\strut \partial t} = 2a^2 \dfrac{\strut \partial^2 u}{\strut \partial r^2} + \dfrac{\strut \alpha}{\strut cl}(u_c - u), r = 0 \\
	u(r, 0) = u_b + J_0\left(\dfrac{\strut \mu_1 r}{\strut R}\right), 0 \leq r \leq R \\
	u(R, t) = u_b, 0 \leq t \leq T
	\end{array}
	\right.
	\end{equation}

	\subsection{Построение аналитического решения}
	Рассмотрим систему \eqref{KoshiTask} в следующем виде:
	\begin{equation}\label{KoshiTaskTestSolution}
		\left \{\begin{array}{l}
		\dfrac{\partial u}{\partial t} = a^2 \left ( \dfrac{1}{r}\dfrac{\partial u}{\partial r} + \dfrac{\partial^2 u}{\partial r^2}\right ) + \dfrac{\alpha}{cl}(u_b - u), 0 < r \leq R\\
		\\
		u(r, 0) = u_b + J_0\left(\dfrac{\mu_1 r}{R}\right), 0 \leq r \leq R \\
		\\
		u(R, t) = u_b
		\end{array}\right.
	\end{equation}
	
	В этом уравнении $J_0 \left ( \dfrac{\mu_1 r}{R} \right)$ --- функция Бесселя нулевого порядка, а $\mu_1$ --- первый корень уравнения $J \left( \mu \right) = 0$.
	
	Воспользуемся вспомогательной задачей на собственные значения уравнения Лапласа. Для начала, сделаем так чтобы граничные условия были однородными, для этого осуществим замену:
	\begin{equation*}
	u = v + u_b
	\end{equation*}
	
	Тогда задача \eqref{KoshiTaskTestSolution} будет выглядеть следующим образом
	\begin{equation}\label{KoshiTaskTestSolutionUniform}
	\left \{\begin{array}{l}
	\dfrac{\partial v}{\partial t} = a^2 \left ( \dfrac{1}{r}\dfrac{\partial v}{\partial r} + \dfrac{\partial^2 v}{\partial r^2}\right ) - \dfrac{\alpha}{cl}v, 0 < r \leq R\\
	\\
	v(r, 0) = J_0\left(\dfrac{\mu_1 r}{R}\right), 0 \leq r \leq R \\
	\\
	v(R, t) = 0
	\end{array}\right.
	\end{equation}
	
	Теперь рассмотрим следующую спектральную задачу:
	\begin{equation*}
	\hat{L}v = \lambda v
	\end{equation*}
	\begin{equation*}
	\dfrac{1}{r}\dfrac{\partial v}{\partial r} + \dfrac{\partial^2 v}{\partial r^2} = \lambda v
	\end{equation*}
	
	Это уравнение можно привести к уравнению Бесселя и записать его решение в следующем виде:
	\begin{equation*}
		v = C(t)J_0\left( \sqrt{-\lambda}r \right)
	\end{equation*}
	
	Решение системы \eqref{KoshiTaskTestSolutionUniform} можно записать в виде бесконечного ряда по собственным функциям оператора Лапласа:
	\begin{equation}\label{LaplaceSolutionSeries}
		v(r,t) = \sum\limits^{\infty}_{i=0} C_i(t)J_0\left(\sqrt{-\lambda_i}r\right)
	\end{equation}
	
	Рассмотрим граничное условие $v(R,t) = 0$:
	\begin{equation*}
		v(R,t) = \sum\limits^{\infty}_{i=0} C_i(t)J_0\left(\sqrt{-\lambda_i}R\right) = 0 \Rightarrow J_0\left(\sqrt{-\lambda_i}R\right) = 0 \Rightarrow \lambda_i = -\dfrac{\mu_i^2}{R^2}
	\end{equation*}
	
	Теперь рассмотрим другое граничное условие $v(r,0) = J_0\left(\dfrac{\mu_1 r}{R}\right)$:
	\begin{equation*}
		v(r,0) = \sum\limits^{\infty}_{i=0} C_i(0)J_0\left(\dfrac{\mu_i r}{R}\right) = J_0\left(\dfrac{\mu_1 r}{R}\right) \Rightarrow C_i(0) = \left\{\begin{array}{c}
			1, i = 1 \\
			0, i \ne 1
		\end{array}\right.
	\end{equation*}
	
	Подставим разложение \eqref{LaplaceSolutionSeries} в первое уравнение системы \eqref{KoshiTaskTestSolutionUniform}
	\begin{equation*}\label{key}
	\dfrac{\partial C_i}{\partial t} = - \left ( \dfrac{\mu_i^2}{R^2}a^2 + \dfrac{\alpha}{cl} \right) C_i = -\gamma_i C_i \Rightarrow C_i = A_i \exp{\left( -\gamma_i t \right)}
	\end{equation*}
	
	Подставим граничные условия для $C_i(t)$, получим
	\begin{equation*}
		A_i = \left\{\begin{array}{c}
		1, i = 1 \\
		0, i \ne 1
		\end{array}\right.
	\end{equation*}
	
	Тогда решение системы \eqref{KoshiTaskTestSolutionUniform} запишется в виде:
	\begin{equation*}
		v(r,t) = \exp{\left[ \left ( \dfrac{\mu_1^2}{R^2}a^2 + \dfrac{\alpha}{cl} \right)t \right]}J_0 \left( \dfrac{\mu_1 r}{R}\right)
	\end{equation*}
	
	Решение уравнения $J_0(\mu) = 0$, найдем при помощи математического пакета $\mu_1 = 2.4048255$.
	\subsection{Построение разностной схемы}
	
	Для построения разностной схемы для начала введем равномерную сетку с шагами $h_r$ --- пространственный шаг и $h_t$ --- шаг по времени:	
	\begin{equation}
	\begin{array}{l}
	r_i = ih_r, \quad i = \overline{0,I} \quad \\                         
	t_k = kh_t, \quad k = \overline{0,K}
	\end{array}
	\label{eq:grid}
	\end{equation}
	
	Значения шагов равномерной сетки можно найти из следующих соотношений:	
	\begin{equation*}
	h_r = \dfrac{R}{I}
	\end{equation*}	
	\begin{equation*}
	h_t = \dfrac{T}{K}
	\end{equation*}

	И рассмотрим 4 разностных отношения для производных по времени и пространству:	
	\begin{equation*}
	\left.\dfrac{\partial u}{\partial t}\right|_{(r_i,t_k)} = \dfrac{u_i^k-u_i^{k-1}}{h_t}, \quad i = \overline{1, I-1}, k = \overline{1,K}
	\end{equation*}	
	\begin{equation*}
	\left.\dfrac{\partial u}{\partial r}\right|_{(r_i, t_k)} = \dfrac{u_{i+1}^k - u_{i-1}^k}{2h_r}, \quad i = \overline{1,I-1}, k = \overline{1,K}
	\end{equation*}	
	\begin{equation*}
	\left.\dfrac{\partial^2 u}{\partial r^2}\right|_{(r_i, t_k)} = \dfrac{u_{i+1}^k - 2u_i^k + u_{i-1}^k}{h^2_r}, \quad i = \overline{1,I-1}, k = \overline{0,K}
	\end{equation*}
	На основе этих соотношений построим неявную конечно-разностную схему:
	\begin{equation*}
		\left\{\begin{array}{l}
			\dfrac{u_i^{k}-u_i^{k-1}}{h_t} = a^2\left(\dfrac{1}{ih_r}\dfrac{u_{i+1}^k - u_{i-1}^k}{2h_r} +\dfrac{u_{i+1}^k - 2u_i^k + u_{i-1}^k}{h^2_r}\right) + \dfrac{\alpha}{cl}\left (u_c - u_i^k \right) \\
			i = \overline{1,I-1}, \quad k = \overline{0,K-1}	 \\
			\dfrac{u_0^{k}-u_0^{k-1}}{h_t} = 2a^2\dfrac{u_1^k-2u_0^k+u_{-1}^k}{h_r^2} + \dfrac{\alpha}{cl}(u_c-u_0^k) \\
			u_i^0 = u_b + J_0\left( \dfrac{\mu_1 ih_r}{R} \right) \\
			i = \overline{0, I-1} \\
			u_I^k = u_b \\
		k = \overline{0,K}
	\end{array}\right.
	\end{equation*}
	В связи с радиальной симметрией запишем соотношение для первой пространственной производной:
	\begin{equation*}
		\dfrac{u_1 - u_{-1}}{h_r} = 0
	\end{equation*}
	Из этого соотношения получим:
	\begin{equation*}
		u_{-1} = u_1
	\end{equation*}
	В соответствии с этим получим:
	\begin{equation}\label{ImplicitScheme}
		\left\{\begin{array}{l}
			\dfrac{u_i^{k}-u_i^{k-1}}{h_t} = a^2\left(\dfrac{1}{ih_r}\dfrac{u_{i+1}^k - u_{i-1}^k}{2h_r} +\dfrac{u_{i+1}^k - 2u_i^k + u_{i-1}^k}{h^2_r}\right) + \dfrac{\alpha}{cl}\left (u_c - u_i^k \right) \\
			i = \overline{1,I-1}, \quad k = \overline{0,K-1}	 \\
			\dfrac{u_0^{k}-u_0^{k-1}}{h_t} = 2a^2\dfrac{u_1^k-u_0^k}{h_r^2} + \dfrac{\alpha}{cl}(u_c-u_0^k) \\
			u_i^0 = u_b + J_0\left( \dfrac{\mu_1 ih_r}{R} \right) \\
			i = \overline{0, I-1} \\
			u_I^k = u_b \\
			k = \overline{0,K}
		\end{array}\right.
	\end{equation}
	
	Перепишем первое уравнение этой схемы в виде, удобном для решения дискретной задачи:
	\begin{equation*}
		Au^k_{i-1} + Bu^k_i + Cu^k_{i+1} = Du_i^{k-1} + E
	\end{equation*}
	
	Опуская промежуточные выкладки, запишем значения коэффициентов $A$, $B$, $C$, $D$, $E$, $F$:
	\begin{equation*}
		A = -\dfrac{a^2h_t}{2ih_r} + \dfrac{a^2h_t}{h_r^2}
	\end{equation*}
	\begin{equation*}
		B = 1 - \dfrac{2h_ta^2}{h_r^2} - \dfrac{\alpha h_t}{cl}
	\end{equation*}
	\begin{equation*}
		C = \dfrac{a^2h_t}{2ih_r^2} + \dfrac{a^2h_t}{h_r^2}
	\end{equation*}
	\begin{equation*}
		D = -1
	\end{equation*}
	\begin{equation*}
		E = - \dfrac{\alpha u_c h_t}{cl}
	\end{equation*}
	
	Аналогично представим второе уравнение этой системы и коэффициенты $F$, $G$, $H$:
	\begin{equation*}
		u_0^k = Fu_0^{k-1} + Gu_1^k + H
	\end{equation*}
	\begin{equation*}
		F = \dfrac{1}{1 + \dfrac{2a^2h_t}{h_r^2} + \dfrac{\alpha h_t}{cl}}
	\end{equation*}
	\begin{equation*}
		G = \dfrac{\dfrac{2a^2h_t}{h_r}}{1 + \dfrac{2a^2h_t}{h_r^2} + \dfrac{\alpha h_t}{cl}}
	\end{equation*}
	\begin{equation*}
		H = \dfrac{\dfrac{\alpha h_t}{cl}u_1^k}{1 + \dfrac{2a^2h_t}{h_r^2} + \dfrac{\alpha h_t}{cl}}
	\end{equation*}
	\newpage
	\subsection{Исследование аппроксимации и устойчивости разностной схемы}
	Систему уравнений \eqref{KoshiTask} можно представить в операторном виде:
	\begin{equation}\label{OperatorViewKoshiTask}
 	\hat{L}u = \left\{\begin{array}{l}
 		\hat{L}^1u \\
 		\hat{L}^2u \\
 		\hat{L}^3u \\
 		\hat{L}^4u \\
 	\end{array}
 	\right\} = \left\{\begin{array}{c}
 		\left[\dfrac{\strut\partial}{\strut\partial t} - a^2\left( \dfrac{\strut 1}{\strut r}\dfrac{\strut \partial }{\strut \partial r} + \dfrac{\strut \partial^2}{\strut \partial r^2} + \dfrac{\strut \alpha}{\strut cl} \right)\right]u \\
 		\left[\dfrac{\strut \partial }{\strut \partial t} - 2a^2\dfrac{\strut\partial^2}{\strut\partial r^2} + \dfrac{\strut\alpha}{\strut cl}\right]u\\
 		u\vert_{t=0} \\
 		u\vert_{r=R} \\
 	\end{array}\right\}, f = \left\{\begin{array}{c}
 		\dfrac{\strut \alpha u_c}{\strut cl} \\
 		\dfrac{\strut \alpha u_c}{\strut cl} \\
 		u_c + J_0\left(\dfrac{\mu_1 r}{R}\right) \\
 		u_c
 	\end{array}\right\}
	\end{equation}
 	Конечно -- разностная задача \eqref{ImplicitScheme} можно также представить в операторном виде:
 	\begin{equation}\label{OperatorViewImplicitScheme}
 	\hat{L}_hu_h = \left\{\begin{array}{l}
 	\hat{L}_h^1u \\
 	\hat{L}_h^2u \\
 	\hat{L}_h^3u \\
 	\hat{L}_h^4u \\
 	\end{array}
 	\right\} = \left\{\begin{array}{c}
 	\dfrac{u_i^{k}-u_i^{k-1}}{h_t} - a^2\left(\dfrac{1}{ih_r}\dfrac{u_{i+1}^k - u_{i-1}^k}{2h_r} +\dfrac{u_{i+1}^k - 2u_i^k + u_{i-1}^k}{h^2_r}\right) + \dfrac{\alpha}{cl}u_i^k \\
 	\dfrac{u_0^{k}-u_0^{k-1}}{h_t} - 2a^2\dfrac{u_1^k-u_0^k}{h_r^2} + \dfrac{\alpha}{cl}u_0^k\\
 	u_i^0 \\
 	u_N^k \\
 	\end{array}\right\}
 	\end{equation}
 	
 	\paragraph{Определение порядка аппроксимации.} Рассмотрим первый компонент вектора невязки
 	\begin{eqnarray}
 		\delta f^1_h = \left\{ L^1_h[u]_h - f^1_h \right\} _{(r_i, t_k)} = \dfrac{u(r_i, t_k) - u(r_i, t_{k-1})}{h_t} - \nonumber \\ - a^2\left( \dfrac{1}{ih_r}\dfrac{u(r_{i+1}, t_k) - u(r_{i-1}, t_k)}{2h_r} +\dfrac{u(r_{i+1}, t_k) - 2u(r_i, t_k) + u(r_{i-1}, t_k)}{h^2_r} \right) + \nonumber \\ + \dfrac{\alpha}{cl}u(r_i, t_k) - \dfrac{\alpha u_c}{cl}
 	\end{eqnarray}
 	
 	Воспользуемся разложением Тейлора для функции $u$:
 	\begin{equation*}
	 	u(r_i, t_{k-1}) = u(r_i, t_k) - \dfrac{\partial u(r_i, t_k)}{\partial t}\dfrac{h_t}{1!} + \dfrac{\partial^2 u(r_i, t_k)}{\partial t^2}\dfrac{h_t^2}{2!} + O(h_t^3)
 	\end{equation*}
 	\begin{equation*}
	 	u(r_{i-1}, t_{k}) = u(r_i, t_k) - \dfrac{\partial u(r_i, t_k)}{\partial r}\dfrac{h_r}{1!} + \dfrac{\partial^2 u(r_i, t_k)}{\partial r^2}\dfrac{h_r^2}{2!} + O(h_r^3)
 	\end{equation*}
 	\begin{equation*}
	 	u(r_{i+1}, t_{k}) = u(r_i, t_k) + \dfrac{\partial u(r_i, t_k)}{\partial r}\dfrac{h_r}{1!} + \dfrac{\partial^2 u(r_i, t_k)}{\partial r^2}\dfrac{h_r^2}{2!} + O(h_r^3)
 	\end{equation*}
 	
 	Воспользуемся этими выражениями для $\delta f_h^1$:
 	\begin{eqnarray}
 		\delta f^1_h = \left\{\dfrac{u - u + u_th_t + - \dfrac{1}{2}u_{tt}h_t^2 + O(h_t^3)}{h_t} - a^2 \left( \dfrac{1}{ih_r}\dfrac{1}{2h_r} \left( u + u_rh_r+u_rrh_r^2/2 +  \right. \right. \right. \nonumber \\
 		\left.+ O(h^3) - u + u_rh_r -u_{rr}h_r^2/2 + O(h_r^3)\right) + \dfrac{1}{h_r^2}\left( u + u_rh_r + u_{rr}h_r^2/2 + \right. \nonumber \\ \left.\left. +  u_{rrr}h_r^3/6 + O(h_r^4) - 2u + u - u_r + u_{rr}h_r^2/2 - u_{rrr}h_r^3/6 + O(h_r^4) \right)\right) + \nonumber \\
 		\left.+  \dfrac{2\alpha}{cl}u - \dfrac{2\alpha}{cl}u_c\right\}_{(r_i, t_k)} \nonumber
 	\end{eqnarray}
 	
 	Осуществляем сокращения:
 	\begin{eqnarray}
 		\delta f_h^1 = \left\{ u_t + O(h_t) - a^2 \left( \dfrac{1}{ih_r}(u_r + O(h_r^2)) + u_{rr} + O(h^2) \right)  + \nonumber \right. \\ 
 		\left. +  \dfrac{2\alpha}{cl}u - \dfrac{2\alpha}{cl}u_c \right\}_{(r_i, t_k)} = \left\{ u_t + O(h_t) - a^2 \left( \dfrac{1}{ih_r}(u_r + O(h_r^2)) + u_{rr} + O(h^2) \right) -\right. \nonumber \\ \left. - u_t + a^2 \left( \dfrac{1}{ih_r}u_r + u_{rr} \right)\right\}_{(r_i, t_k)} = \left\{O(h_t) + a^2\left(\dfrac{1}{ih_r} + 1 \right)O(h_r^2)\right\}_{(r_i, t_k)} \nonumber
 	\end{eqnarray}
 	
 	Теперь рассмотрим второй компонент вектора невязки:
 	\begin{eqnarray}
	 	\delta f^2_h = \left\{ L_h^2[u]_h - f^2_h \right\}_{(r_0, t_k)} = \dfrac{u(r_0, t_k) - u(r_0, t_{k-1})}{h_t} - 2a^2\dfrac{u(r_1, t_k) - u(r_0, t_k)}{h_r^2} + \nonumber \\
	 	+ \dfrac{2\alpha}{cl}u(t_k, r_0) - \dfrac{2\alpha}{cl}u_c \nonumber
 	\end{eqnarray}
 	
 	Аналогично первому компоненту невязки воспользуемся соотношениями для разложения в ряд Тейлора:
 	\begin{eqnarray}
 		\delta f^2_h = \left\{ \dfrac{u - u + u_t - O(h_t^2)}{h_t} - 2a^2\dfrac{u+u_rh_r + u_{rr}h_r^2/2 + u_{rrr}h_r^3/6  + O(h_r^4) - u}{h_r^2} + \nonumber \right.  \\ 
 		\left.  + \dfrac{2\alpha}{cl}u - \dfrac{2\alpha}{cl}u_c\right\}_{(r_0, t_k)} \nonumber
 	\end{eqnarray}
 	
 	Таким же образом осуществляем сокращения:
 	\begin{eqnarray}
	 	\delta f^2_h = \left\{ u_t + O(h_t) - 2a^2 \left(\dfrac{u_rh_r + u_{rr}h_r^2/2 + u_{rrr}h_r^3/6+O(h_r^4)}{h_r^2} \right) + \dfrac{2\alpha u}{cl} \right. \nonumber \\
	 	\left. - u_t + 2a^2u_rr - \dfrac{2\alpha u}{cl} \right\}_{(r_0, t_k)} = \left\{ O(h_t) + 2a^2O(h_r^2)\right\}_{(r_0, t_k)} \nonumber
 	\end{eqnarray}
 	
 	Для третьей и четвертой компонент невязки получим:
 	\begin{equation*}
	 	\delta f^3_h = \left\{ L_h^3[u]_h - f^3_h \right\}_{(r_i, t_0)} = u(r_i, t_0) - u(r_i, t_0) = 0
 	\end{equation*}
 	\begin{equation*}
	 	\delta f^4_h = \left\{ L_h^2[u]_h - f^2_h \right\}_{(r_I, t_k)} = u(r_I, t_k) - u(r_I, t_k) = 0
 	\end{equation*}
 	
 	Таким образом дискретная задача \eqref{ImplicitScheme} аппроксимирует непрерывную задачу \eqref{KoshiTask} с первым порядком по времени и вторым порядком по координате.
	\chapter{Лабораторная работа №2}
	\section{Тема работы. Цель работы. Порядок выполнения работы}
	\section{Описание этапов выполнения лабораторной работы}
	\subsection{Описание тестовой задачи, ее аналитическое решение}
	\subsection{Экспериментальное исследование фактической скорости сходимости сеточного решения к точному для тестовой задачи}
\end{document}